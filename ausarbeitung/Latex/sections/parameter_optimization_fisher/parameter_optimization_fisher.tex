\section{Optimierung der Parameter des Fisher-Algorithmus}
Um einen höheren AMS-Wert zu erreichen, variierten wir folgende zwei Parameter des Fisher-Algorithmus.
\begin{equation}
\begin{split}
 \mbox{\textit{PDFInterpolMVAPdf}} &\in \{ \mbox{ Spline1,  Spline2, Spline3}  \}  \\
 \mbox{\textit{NsmoothMVAPdf}} &\in \{ \mbox{5,10,15, \ldots, 195} \}
\end{split}
\end{equation}
Der erste Parameter steht für den Grad der Splines mit denen die Wahrscheinlichkeitsverteilungen der Eingabeparameter interpoliert werden. Der zweite Parameter steht für die Anzahl an iterativen Glättungen dieser Verteilungen. 
Das Training dieser Methode mit dem gegebenen Datensatz kostet wenig Zeit - meistens weniger als Minute. Das macht es möglich mit einem brute-force-Ansatz alle Parameterkonstellationen durchzuprobieren. Der AMS-Wert war unabhängig von der Wahl der Parameter \textit{PDFInterpolMVAPdf} und \textit{NsmoothMVAPdf} konstant 0.496. Wir verzichteten auf weitere Optimierungen unter Verwendung des Fisher-Algorithmus, da die erzielten AMS-Werte deutlich unter denen der BDT-Methode lagen.